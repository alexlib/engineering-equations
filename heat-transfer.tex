% Copyright (C) 2013, 2014  Jim Turner
%
% This work is licensed under the Creative Commons Attribution-ShareAlike 3.0 Unported License. To
% view a copy of this license, visit http://creativecommons.org/licenses/by-sa/3.0/deed.en_US.

\documentclass{article}
\title{Heat Transfer Equations}

% Copyright (C) 2013  Jim Turner
%
% This work is licensed under the Creative Commons Attribution-ShareAlike 3.0 Unported License. To
% view a copy of this license, visit http://creativecommons.org/licenses/by-sa/3.0/deed.en_US.

% Page margins
\usepackage[margin=1in]{geometry}

% Muticolumn
\usepackage{multicol}

% Subcaptions
\usepackage{subcaption}

% Better tables
\usepackage{longtable}
\usepackage{tabu}
\usepackage{booktabs}

% Mathematics
\usepackage{amsmath}

% Bold math symbols (e.g. vectors)
\usepackage{bm}

% Differential operators and derivatives
\usepackage{commath}

% Numbers and units
\usepackage{siunitx}
\sisetup{per-mode = symbol,
  inter-unit-product = \ensuremath{{}\cdot{}}}

% Misc LaTeX patches
\usepackage{fixltx2e}

% Compact list environments
\usepackage{mdwlist}

% Title without author and date
\makeatletter
\renewcommand{\@maketitle}{\newpage\null\begin{center}{\LARGE \@title \par}\end{center}\par}
\makeatother

% Description environment labels are not bold and are followed by colon
\renewcommand*\descriptionlabel[1]{\hspace\labelsep\normalfont #1:}

% Bold vectors
\renewcommand{\vec}[1]{\bm{#1}}
\newcommand{\uvec}[1]{\bm{\hat{#1}}}

% Del operator
\renewcommand{\del}{\vec{\nabla}}

% Volume
\newcommand{\volsym}{\settowidth{\dimen255}{V}\ooalign{$V$\cr\rule[.7ex]{\dimen255}{.4pt}\cr}}

% Error function and its complement
\DeclareMathOperator\erf{erf}
\DeclareMathOperator\erfc{erfc}

% Copyright notice on first page
\usepackage{fancyhdr}
\renewcommand{\headrulewidth}{0pt}

% Hyperlinks and PDF document title
\usepackage[pdftex]{hyperref}
\hypersetup{colorlinks,citecolor=black,filecolor=black,linkcolor=black,urlcolor=black}

\setcounter{tocdepth}{1}
\usepackage{amssymb}

\begin{document}

% Copyright (C) 2013-2015  Jim Turner
%
% This work is licensed under the Creative Commons Attribution-ShareAlike 4.0 International License.
% To view a copy of this license, visit https://creativecommons.org/licenses/by-sa/4.0/.

\fancypagestyle{plain}{
\fancyfoot[C]{Copyright \textcopyright\ 2013--2015 Jim Turner. This work is licensed under the Creative
  Commons Attribution-ShareAlike~4.0 International License. To view a copy of this license, visit
  \mbox{\url{https://creativecommons.org/licenses/by-sa/4.0/}}. The \LaTeX{} source files are
  available under the same license at \mbox{\url{https://github.com/jturner314/engineering-equations}}.}
}

\maketitle
\tableofcontents

\section*{Thanks}

The references for the equations are listed in the References section.  Thanks also to those
students who used this equations sheet and provided feedback.  This feedback was very helpful for
finding errors in equations, discovering missing equations, etc.

\section*{Contributing}

This document and its sources are available under a free license (see the footer) in the hopes that
you will find them useful and so that you have the freedom to make them even better.  All
development for this equation sheet and others is done on GitHub.  If you'd like to contribute,
please visit the project page at the URL provided in the footer.  If you find an error or other area
for improvement, please create an issue or send a pull request via GitHub.

\newpage


\begin{multicols}{2}
\setlength{\columnseprule}{0.5pt}

\section{Notation}
\begin{tabu}{lcc}
  \toprule
  Meaning & Units & Symbol \\
  \midrule
  Heat & \si{\joule} & $Q$ \\
  Heat rate & \si{\watt} & $q$ \\
  Heat flux & \si{\watt\per\square\meter} & $q''$ \\
  Volumetric heat generation & \si{\watt\per\cubic\meter} & $\dot q$ \\
  \bottomrule
\end{tabu}

\section{Basic Heat Transfer}

\begin{description*}
\item[Conduction (Fourier's law)]
  \(\vec{q''} = -k \del T\)
\item[Convection (Newton's law of cooling)]~\\
  \(q'' = h(T_\mathrm{s}-T_\infty)\)
\item[Radiation] See Section \ref{sec:radiation}
\item[First Law of Thermodynamics]
  \(\dot{E}_\mathrm{in}-\dot{E}_\mathrm{out}+\dot{E}_\mathrm{g} = \dot{E}_\mathrm{st}\)
\end{description*}

\section{Heat Equation}
\begin{description*}
\item[Differential equations]~
  \begin{description*}
  \item[Rectangular]
    \(\rho c_p \pd{T}{t}
    = \dot q
    + \pd{}{x}\left(k\pd{T}{x}\right) \\
    + \pd{}{y}\left(k\pd{T}{y}\right)
    + \pd{}{z}\left(k\pd{T}{z}\right)\)
  \item[Cylindrical]
    \(\rho c_p \pd{T}{t}
    = \dot q
    + \frac{1}{r}\pd{}{r}\left(kr\pd{T}{r}\right) \\
    + \frac{1}{r^2}\pd{}{\phi}\left(k\pd{T}{\phi}\right)
    + \pd{}{z}\left(k\pd{T}{z}\right)\)
  \item[Spherical]
    \(\rho c_p \pd{T}{t}
    = \dot q
    + \frac{1}{r^2}\pd{}{r}\left(kr^2\pd{T}{r}\right) \\
    + \frac{1}{r^2\sin^2\theta}\pd{}{\phi}\left(k\pd{T}{\phi}\right)
    + \frac{1}{r^2\sin\theta}\pd{}{\theta}\left(k\sin\theta\pd{T}{\theta}\right)\)
  \end{description*}
\item[1-D steady-state boundary conditions]~
  \begin{description*}
  \item[Known surface temperature]
    \(T(0,t) = T_\mathrm{s}\)
  \item[Known heat flux]
    \(\left.-k\pd{T}{x}\right|_{x=0} = q''_\mathrm{s}\)
  \end{description*}
\item[1-D steady-state solutions]~
  \begin{description*}
  \item[No generation] See Table \ref{tab:1-d-solutions}
  \item[Plane wall of width $2L$ with uniform $\dot q$]
    \(T(x) \\
    = \frac{\dot{q}L^2}{2k}\left(1-\frac{x^2}{L^2}\right)
    + \frac{T_\mathrm{s,2}-T_\mathrm{s,1}}{2L}\frac{x}{L}
    + \frac{T_\mathrm{s,1}+T_\mathrm{s,2}}{2}\)
  \item[Uniform Joule heating]
    \(\dot q = I^2R_\mathrm{e}/\volsym\)
  \item[Other conditions] See Appendix C of \cite{hamt}
  \end{description*}
\end{description*}

\section{Thermal Resistances}

\begin{description*}
\item[Definition]
  \(R_\mathrm{t} \equiv \Delta{}T/q\);
  the following are also used for transfer per unit area\slash{}length:
  \(R''_\mathrm{t} = RA\) and \(R'_\mathrm{t} = RL\)
\item[Conduction] See Table \ref{tab:1-d-solutions}
\item[Convection]
  \(R_\mathrm{t,conv} = \frac{1}{hA}\)
\item[Radiation]
  \(R_\mathrm{t,rad} = \frac{1}{h_rA}\)
\item[Contact]
  \(R_\mathrm{t,c}\)
\item[Overall heat transfer coefficient, $U$]
  \(q_x = UA\Delta{}T_\mathrm{overall}\); note that
  \(R_\mathrm{tot} = \frac{1}{UA}\)
\end{description*}

\end{multicols}

\begin{table}[htb]
  \centering
  \caption{One-dimensional, steady-state solutions to the heat equation with no
    generation~\cite{hamt}}
  \label{tab:1-d-solutions}
  \tabulinesep=_3pt^3pt
  \begin{tabu}[c]{l c c c}
    \toprule
    & Plane Wall & Cylindrical Wall & Spherical Wall \\
    \midrule
    Heat equation
    & \(\od[2]{T}{x} = 0\)
    & \(\frac{1}{r}\od{}{r}\left(r\od{T}{r}\right) = 0\)
    & \(\frac{1}{r^2}\od{}{r}\left(r^2\od{T}{r}\right) = 0\) \\
    Temperature distribution
    & \(T_\mathrm{s,1} - \Delta{}T\frac{x}{L}\)
    & \(T_\mathrm{s,2} + \Delta{}T\left[\frac{\ln(r/r_2)}{\ln{(r_1/r_2)}}\right]\)
    & \(T_\mathrm{s,1} - \Delta{}T\left[\frac{1-(r_1/r)}{1-(r_1/r_2)}\right]\) \\
    Critical radius of insulation, $r_\mathrm{cr}$
    & none
    & \(\frac k h\)
    & \(\frac{2k}{h}\) \\
    Heat rate, $q$
    & \(kA\frac{\Delta{}T}{L}\)
    & \(\frac{2\pi{}Lk\Delta{}T}{\ln(r_2/r_1)}\)
    & \(\frac{4\pi{}k\Delta{}T}{(1/r_1)-(1/r_2)}\) \\
    Thermal resistance, $R_\mathrm{t,cond}$
    & \(\frac{L}{kA}\)
    & \(\frac{\ln(r_2/r_1)}{2\pi{}Lk}\)
    & \(\frac{(1/r_1)-(1/r_2)}{4\pi{}k}\) \\
    \bottomrule
  \end{tabu}
\end{table}

\begin{multicols}{2}
\setlength{\columnseprule}{0.5pt}

\section{Fins}
\label{sec:fins}

\begin{description*}
\item[Uniform fin]
  \(\od[2]{\theta}{x} - m^2\theta = 0\) where
  \(\theta \equiv T-T_\infty\),
  \(m^2 \equiv \frac{hP}{kA_\mathrm{c}}\),
  $P$ is perimeter, and
  $A_\mathrm{c}$ is cross-sectional area
  \begin{description*}
  \item[Solution]
    \(\theta = C_1e^{mx} + C_2e^{-mx}\)
  \item[Boundary conditions] See Table 3.4 of \cite{hamt}
  \end{description*}
\item[Fin efficiency]
  \(\eta_f \equiv \frac{q_\mathrm{f}}{q_\mathrm{f,max}}
  = \frac{q_\mathrm{f}}{hA_\mathrm{f}\theta_\mathrm{b}}\)
  where $A_\mathrm{f}$ is the surface area of the fin excluding the base
\item[Fin effectiveness]
  \(\varepsilon_\mathrm{f} \equiv \frac{q_\mathrm{f}}{hA_\mathrm{c,b}\theta_\mathrm{b}}
  = \frac{R_\mathrm{t,b}}{R_\mathrm{t,f}}\)
\item[Fin resistance]
  \(R_\mathrm{t,f} \equiv \frac{\theta_\mathrm{b}}{q_\mathrm{f}}
  = \frac{1}{hA_\mathrm{f}\eta_\mathrm{f}}\)
\item[Infinite fin]
  \(\eta_\mathrm{f} = 0\) and
  \(\varepsilon_\mathrm{f} = \sqrt{\frac{kP}{hA_\mathrm{c}}}\)
\item[Fin array]~
  \begin{description*}
  \item[Total surface area]
    \(A_\mathrm{t} = NA_\mathrm{f} + A_\mathrm{b}\)
  \item[Overall surface efficiency]
    \(\eta_\mathrm{o} = 1 - \frac{NA_\mathrm{f}}{A_\mathrm{t}}(1-\eta_\mathrm{f})\)
  \item[Overall resistance]
    \(R_\mathrm{t,o} = \frac{\theta_\mathrm{b}}{q_\mathrm{t}}
    = \frac{1}{\eta_\mathrm{o}hA_\mathrm{t}}\)
  \end{description*}
\end{description*}

\section{Energy Balance Method for Nodal Analysis}
\begin{description*}
\item[Method] For each node, assume all heat flows in, then
  \(\dot{E}_\mathrm{in} + \dot{E}_\mathrm{g} = 0 \\
  \implies \sum_{i=1}^4q_{(i)\rightarrow(m,n)} + \dot{q}(\Delta{}x\cdot\Delta{}y\cdot{}l) = 0\)
\item[E.g. for conduction from $(m-1,n)$ to $(m,n)$]
  \(q_{(m-1,n)\rightarrow(m,n)} = k(\Delta{}y\cdot{}l)\frac{T_{m-1,n}-T_{m,n}}{\Delta{}x}\)
\end{description*}

\section{Transient Conduction}

\subsection{Lumped Capacitance}
\begin{description*}
\item[Biot number] Lumped capacitance method assumes uniform temperature throughout the material;
  this is valid when the Biot number \(\mathit{Bi} \ll 1\).
  \begin{description*}
  \item[Definition]
    \(\mathit{Bi} \equiv hL_\mathrm{c}/k\) where \(L_\mathrm{c}\) is the characteristic length,
    usually defined as \(\volsym / A_\mathrm{s}\).
  \item[Physical interpretation]~\\
    \(\mathit{Bi} = \frac{L_\mathrm{c}/(kA_\mathrm{s})}{1/(hA_\mathrm{s})}
    \approx R_\mathrm{cond}/R_\mathrm{conv}\)
  \end{description*}
\item[General lumped capacitance method]
  \(\rho \volsym c \od{T}{t} \\
  = q''_\mathrm{s}A_\mathrm{s,h} - hA_\mathrm{s,c}(T-T_\infty)
  - h_\mathrm{r}A_\mathrm{s,r}(T-T_\mathrm{sur}) + \dot{E}_\mathrm{g}\)
  for applied heat flux \(q''_\mathrm{s}\) on \(A_\mathrm{s,h}\), convection on \(A_\mathrm{s,c}\),
  radiation on \(A_\mathrm{s,r}\), and generation \(\dot{E}_\mathrm{g}\).
\item[Thermal time constant]
  \(\tau_\mathrm{t} \equiv R_\mathrm{t}C_\mathrm{t}
  = \left(\frac{1}{hA_\mathrm{s}}\right)(\rho\volsym{}c)\)
\item[With radiation and \(T_\mathrm{sur}=\SI{0}{\kelvin}\)]
  \(t = \frac{\rho\volsym{}c}{3\varepsilon\sigma A}\left(\frac{1}{T^3}-\frac{1}{T_\mathrm{i}^3}\right)\)
\item[Negligible radiation]~
  \begin{description*}
  \item[Differential equation]
    \(\od{\theta}{t} + a\theta - b = 0\) where
    \(\theta \equiv T-T_\infty\),
    \(a \equiv 1/\tau_\mathrm{t}\), and
    \(b \equiv \frac{q''_\mathrm{s}A_\mathrm{s,h}+\dot{E}_\mathrm{g}}{\rho\volsym{}c}\)
  \item[Solution]
    \(\frac{\theta-b/a}{\theta_\mathrm{i}-b/a} = e^{-at}\) or, expanded, \\
    \(\frac{\theta}{\theta_\mathrm{i}} = \frac{T-T_\infty}{T_\mathrm{i}-T_\infty}
    = e^{-at} + \frac{b/a}{T_\mathrm{i}-T_\infty}\left(1-e^{-at}\right)\)
  \item[Total energy loss from convection when $q''_\mathrm{s}=0$, $\dot{q}=0$]
    \(Q(t) = \int_0^tq\dif{t}
    = hA_\mathrm{s}\int_0^t\theta\dif{t} \\
    = \rho\volsym{}c\theta_\mathrm{i}(1-e^{-at})\)
  \end{description*}
\end{description*}

\subsection{Semi-Infinite Solid}
\begin{description*}
  \item[Thermal diffusivity]
    \(\alpha = \frac{k}{\rho c}\)
  \item[Constant surface temperature]~\\
    \(T(0,t) = T_\mathrm{s} \neq T(x,0) = T_\mathrm{i} \\
    \implies \frac{T(x,t)-T_\mathrm{s}}{T_\mathrm{i}-T_\mathrm{s}}
    = \erf\left(\frac{x}{2\sqrt{\alpha t}}\right) \\
    \implies q''_\mathrm{s}(t) = \frac{k(T_\mathrm{s}-T_\mathrm{i})}{\sqrt{\pi\alpha t}}\)
  \item[Constant surface heat flux]
    \(q''_\mathrm{s} = q''_0 \\
    \implies T(x,t)-T_\mathrm{i}
    = \frac{2q''_0\sqrt{\alpha t/\pi}}{k}\exp\left(\frac{-x^2}{4\alpha t}\right) \\
    - \frac{q''_0x}{k}\erfc\left(\frac{x}{2\sqrt{\alpha t}}\right)\)
  \item[Constant surface convection]~\\
    \(\left.-k\pd{T}{x}\right|_{x=0} = h[T_\infty-T(0,t)]
    \implies \frac{T(x,t)-T_\mathrm{i}}{T_\infty-T_\mathrm{i}} \\
      = \erfc\left(\frac{x}{2\sqrt{\alpha t}}\right)
      - \left[\exp\left(\frac{hx}{k}+\frac{h^2\alpha t}{k^2}\right)\right]
      \left[\erfc\left(\frac{x}{2\sqrt{\alpha t}}+\frac{h\sqrt{\alpha t}}{k}\right)\right]\)
\end{description*}

\end{multicols}

\section{Radiation}
\label{sec:radiation}

\begin{description*}
  \item[Surface properties]
    \begin{tabu}{lcl}
      \toprule
      Name & Symbol & Meaning \\
      \midrule
      Emissivity & $\varepsilon$ & Ratio of amt. of actual emission to that of blackbody \\
      Absorptivity & $\alpha$ & Portion of irradiation that is absorbed \\
      Reflectivity & $\rho$ & Portion of irradiation that is reflected \\
      Transmissivity & $\tau$ & Portion of irradiation that is transmitted \\
      \bottomrule
    \end{tabu}
  \item[Other definitions]
    \begin{tabu}{lll}
      \toprule
      Flux (\si{\watt\per\square\meter}) & Meaning & Equation \\
      \midrule
      Irradiation & Incident flux & \(G = (\rho+\alpha+\tau)G\) \\
      Emission & Emitted flux & \(E = \varepsilon\sigma{}T^4_\mathrm{s}\) \\
      Radiosity & Flux leaving the surface & When opaque, \(J = E + \rho{}G\)\\
      Net radiative flux & Net flux leaving surface & When opaque,
      \(q''_\mathrm{rad} = \varepsilon\sigma{}T^4_\mathrm{s}-\alpha{}G\) \\
      \bottomrule
    \end{tabu}
  \item[Stefan-Boltzmann constant]
    \(\sigma = \SI{5.67e-8}{\watt\per\square\meter\per\kelvin\tothe{4}}\)
  \item[Solid angle]
    \(\dif{\omega} \equiv \frac{\dif{A_\mathrm{n}}}{r^2}\) where
    $A_\mathrm{n}$ is the area normal to $\vec{r}$
  \item[Spectral intensity]
    \(I_{\lambda,\mathrm{e}}(\lambda,\theta,\phi)
    \equiv \frac{\dif q}{(\dif{A_1}\cos\theta)\cdot\dif{\omega}\cdot\dif{\lambda}}\)
  \item[Heat rate due to radiation]
    \(q_{1\rightarrow2} = I_1A_1\cos\theta_1 \cdot \omega_{2\rightarrow1}\) where
    \(\omega_{2\rightarrow1} = \frac{A_2\cos\theta_2}{r^2}\)
  \item[For diffuse surface]
    total intensity \(I_\mathrm{e} = \textrm{constant}\) and
    total emissive power \(E = \pi{}I_\mathrm{e}\)
  \item[For gray surface ($\alpha=\varepsilon$)]
    \(q''_\mathrm{rad} = \varepsilon E_\mathrm{b} - \alpha G
    = \varepsilon\sigma(T_\mathrm{s}^4-T_\mathrm{sur}^4)\)
  \item[For blackbody] $\varepsilon=1$ and $\alpha=1$
  \item[Radiation heat transfer coefficient, $h_r$]
    \(q''_\mathrm{rad} = h_r(T_\mathrm{s}-T_\mathrm{sur})\) where
    \(h_r = \varepsilon\sigma(T_\mathrm{s}+T_\mathrm{sur})(T_\mathrm{s}^2+T_\mathrm{sur}^2)\)
  \item[Planck distribution] See Table 12.2 of \cite{hamt}
\end{description*}

\section{Convection}
\label{sec:convection}

\emph{Note that many of the equations listed in this section are empirical approximations. For more
  detail about the conditions under which you can apply these equations and for better
  approximations, see \cite{hamt}.}

\begin{description*}
\item[Reynolds number]
  \(\mathit{Re}_\ell = \frac{\rho{}V\ell}{\mu} = \frac{V\ell}{\nu}\)
\item[Prandtl number]
  \(\mathit{Pr} = \frac{c_p\mu}{k} = \frac{\nu}{a}\)
\item[Nusselt number]
  \(\mathit{Nu}_\ell = \frac{h\ell}{k_\mathrm{f}}\)
\item[Drag coefficient]
  \(C_\mathrm{D} \equiv \frac{F_\mathrm{D}}{A_\mathrm{f}(\rho V^2/2)}\)
\end{description*}

\subsection{Flat Plate}
\begin{description*}
\item[Critical Reynolds number]
  \(\mathit{Re}_{x,\mathrm{c}} = \num{5e5}\)
\item[Steady, incompressible flow with constant fluid properties over isothermal plate]~
  \begin{description*}
  \item[Laminar]
    \(\mathit{Nu}_x \equiv \frac{h_xx}{k} = 0.332\mathit{Re}_x^{1/2}\mathit{Pr}^{1/3}\) for
    \(\mathit{Pr} \gtrsim 0.6\)
  \item[Turbulent]
    \(\mathit{Nu}_x = \mathit{St}\mathit{Re}_x\mathit{Pr} = 0.0296\mathit{Re}_x^{4/5}Pr^{1/3}\) for
    \(0.6 \lesssim \mathit{Pr} \lesssim 60\)
  \item[Average]
    \(\overline{\mathit{Nu}}_L = (0.037\mathit{Re}_L^{4/5} - A)\mathit{Pr}^{1/3}\) for
    \(0.6 \lesssim \mathit{Pr} \lesssim 60\) and \(\mathit{Re}_{x,\mathrm{c}} \lesssim Re_L \lesssim 10^8\) where
    \(A = 0.037\mathit{Re}_{x,\mathrm{c}}^{4/5} - 0.664\mathit{Re}_{x,\mathrm{c}}^{1/2}\)
    or \(A = 0\) when fully turbulent
  \end{description*}
\item[Steady, incompressible flow with constant fluid properties over plate with constant heat flux]~
  \begin{description*}
  \item[Laminar]
    \(\mathit{Nu}_x = 0.453\mathit{Re}_x^{1/2}\mathit{Pr}^{1/3}\) for \(\mathit{Pr} \gtrsim 0.6\)
  \item[Turbulent]
    \(\mathit{Nu}_x = 0.0308\mathit{Re}_x^{4/5}\mathit{Pr}^{1/3}\) for \(0.6 \lesssim \mathit{Pr} \lesssim 60\)
  \end{description*}
\end{description*}

\subsection{Cylinder in Cross Flow}
\begin{description*}
\item[Nusselt number]
  \(\overline{\mathit{Nu}}_D
  = 0.3 + \frac{0.62\mathit{Re}_D^{1/2}\mathit{Pr}^{1/3}}
               {[1+(0.4/\mathit{Pr})^{2/3}]^{1/4}}
          \left[1 + \left(\frac{\mathit{Re}_D}{282000}\right)^{5/8}\right]^{4/5}\)
  for \(\mathit{Re}_D\mathit{Pr} \gtrsim 0.2\)
\end{description*}

\subsection{Internal Flow}
\begin{description*}
\item[Critical Reynolds number]
  \(\mathit{Re}_{D,\mathrm{c}} = \num{2300}\)
\item[Heat rate]
  \(q = \dot{m}c_p\Delta{}T\)
\item[Fully developed laminar flow in circular tube]
  \(\mathit{Nu}_D =
  \begin{cases}
    4.36 & q_\mathrm{s}'' = \textrm{constant} \\
    3.66 & T_\mathrm{s} = \textrm{constant}
  \end{cases}\)
\item[Fully developed turbulent flow in smooth circular tube]
  \(\mathit{Nu}_D =
  \begin{cases}
    0.023\mathit{Re}_D^{0.8}\mathit{Pr}^{0.4} & \textrm{for heating} \\
    0.023\mathit{Re}_D^{0.8}\mathit{Pr}^{0.3} & \textrm{for cooling}
  \end{cases}\)
\end{description*}

\subsubsection{Viscous Pipe Flow}
\begin{description*}
\item[Major losses] head losses due to viscous effects in straight pipe sections; given by
  \(h_{L\mathrm{major}} = f\frac{L}{D}\frac{V^2}{2g} = f\frac{L}{D^5}\frac{8Q^2}{\pi^2g}\)
\item[EBE with major losses only]
  \(p_1-p_2 = \rho{}g(z_2-z_1)+\rho{}gh_L = \rho{}g(z_2-z_1) + f\frac{L}{D}\frac{\rho{}V^2}{2}
  = \rho{}g(z_2-z_1) + f\frac{L}{D^5}\frac{8\rho{}Q^2}{\pi^2}\)
\item[Absolute pipe roughness, $\varepsilon$] a measure of the roughness of pipe walls; has units of
  length. Search online or see Figure 8.3 of~\cite{hamt} for values of $\varepsilon$.
\item[Friction factor, $f$] determine flow type with Reynolds number,
  \(\mathit{Re}_D = \frac{\rho{}VD}{\mu} = \frac{4\rho{}Q}{\pi\mu{}D}\), then use:
  \begin{description}
  \item[Laminar]
    \(f = \frac{64}{\mathit{Re}_D}\)
  \item[Transitional]
    highly variable $f$; in general, use turbulent value
  \item[Turbulent]
    \(\frac{1}{\sqrt{f}}
    = -1.8\log\left[\left(\frac{\varepsilon/D}{3.7}\right)^{1.11}
      + \frac{6.9}{\mathit{Re}_D}\right]\)
  \end{description}
\end{description*}

\section{Heat Exchangers}
\label{sec:heat-exchangers}

\begin{description*}
\item[Overall heat transfer coefficient]
  \(\frac{1}{UA} = \frac{1}{(\eta_0hA)_\mathrm{c}} + \frac{R_\mathrm{f,c}''}{(\eta_0A)_\mathrm{c}}
  + R_\mathrm{w} + \frac{R_\mathrm{f,h}''}{(\eta_0A)_\mathrm{h}} + \frac{1}{(\eta_0hA)_\mathrm{c}}\)
  where $R_\mathrm{w}$ is the wall resistance, $\eta_0$ is defined in Section~\ref{sec:fins}, and
  fouling factors $R_\mathrm{f}''$ can be found in Table~11.1 of \cite{hamt}.
\item[First law analysis]
  \(q = \dot{m}_\mathrm{h}(i_\mathrm{h,i} - i_\mathrm{h,o})
  = \dot{m}_\mathrm{c}(i_\mathrm{c,o} - i_\mathrm{c,i})
  = \dot{m}_\mathrm{h}c_{p,\mathrm{h}}(T_\mathrm{h,i} - T_\mathrm{h,o})
  = \dot{m}_\mathrm{c}c_{p,\mathrm{c}}(T_\mathrm{c,o} - T_\mathrm{c,i})\)
\end{description*}

\subsection{Log Mean Temperature Difference (LMTD) Method}
\label{sec:lmtd}

\begin{description*}
\item[First law]
  \(q = UA\Delta T_\mathrm{lm}\) where
  \(\Delta T_\mathrm{lm} = \frac{\Delta T_2 - \Delta T_1}{\ln(\Delta T_2 / \Delta T_1)}
  = \frac{\Delta T_1 - \Delta T_2}{\ln(\Delta T_1 / \Delta T_2)}\),
  \(\Delta T_1 \equiv T_{\mathrm{h},1} - T_{\mathrm{c},1}\), and
  \(\Delta T_2 \equiv T_{\mathrm{h},2} - T_{\mathrm{c},2}\)
\item[Parallel flow heat exchanger]
  \(\Delta T_1 = T_\mathrm{h,i} - T_\mathrm{c,i}\) and
  \(\Delta T_2 = T_\mathrm{h,o} - T_\mathrm{c,o}\)
\item[Counterflow heat exchanger]
  \(\Delta T_1 = T_\mathrm{h,i} - T_\mathrm{c,o}\) and
  \(\Delta T_2 = T_\mathrm{h,o} - T_\mathrm{c,i}\)
\item[Special operating conditions] Depending on the values of $C_\mathrm{h}$ and $C_\mathrm{c}$,
  some simplifications may be made. See Section 11.3.3 of \cite{hamt} for details.
\item[Multipass and cross-flow heat exchangers]
  \(\Delta T_\mathrm{lm} = F \Delta T_\mathrm{lm,CF}\) where $\Delta T_\mathrm{lm,CF}$ is the value
  that would be obtained under the counterflow condition. Values of $F$ can be found in
  Supplement~11S.1 of \cite{hamt}.
\end{description*}

\subsection{Effectiveness--NTU Method}
\label{sec:eff-ntu}

\begin{description*}
\item[Heat capacity rate]
  \(C \equiv \dot{m} c_p\)
\item[Max heat rate]
  \(q_\mathrm{max} = C_\mathrm{min}(T_\mathrm{h,i} - T_\mathrm{c,i})\)
\item[Effectiveness]
  \(\varepsilon \equiv \frac{q}{q_\mathrm{max}}\)
\item[Number of transfer units]
  \(\mathit{NTU} \equiv \frac{UA}{C_\mathrm{min}}\)
\item[Relationship between $\varepsilon$ and $\mathit{NTU}$]
  \(\varepsilon = f_1(\mathit{NTU}, C_\mathrm{r})\) and
  \(\mathit{NTU} = f_2(\varepsilon, C_\mathrm{r})\) where
  \(C_\mathrm{r} \equiv \frac{C_\mathrm{min}}{C_\mathrm{max}}\)
  \begin{description*}
  \item[When \(C_\mathrm{r} = 0\)]
    \(\varepsilon = 1 - e^{-\mathit{NTU}}\) and \(\mathit{NTU} = -\ln(1 - \varepsilon)\)
  \item[Other conditions] See Table 11.3, Table 11.4, and Figures 11.10 through 11.15 of \cite{hamt}.
  \end{description*}
\end{description*}

\begin{thebibliography}{99}
\bibitem{hamt} Bergman, T. L., Lavine, A. S., Incropera, F. P., and Dewitt, D. P., 2011,
  \emph{Fundamentals of Heat and Mass Transfer}, John Wiley \& Sons, Inc., Jefferson City.
\bibitem{klec} Kuznetsov, A. V., 2013, \emph{Heat Transfer Fundamentals}, MAE 310 course lectures,
  North Carolina State University, Raleigh, NC.
\bibitem{Lyons2014} Lyons, K. M., 2014, \emph{Design of Thermal Systems}, MAE 412 course lectures,
  North Carolina State University, Raleigh, NC.
\end{thebibliography}

\end{document}
