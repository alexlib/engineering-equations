% Copyright (C) 2013, 2014  Jim Turner
%
% This work is licensed under the Creative Commons Attribution-ShareAlike 3.0 Unported License. To
% view a copy of this license, visit http://creativecommons.org/licenses/by-sa/3.0/deed.en_US.

\documentclass{article}
\title{Dynamics Equations}

% Copyright (C) 2013  Jim Turner
%
% This work is licensed under the Creative Commons Attribution-ShareAlike 3.0 Unported License. To
% view a copy of this license, visit http://creativecommons.org/licenses/by-sa/3.0/deed.en_US.

% Page margins
\usepackage[margin=1in]{geometry}

% Muticolumn
\usepackage{multicol}

% Subcaptions
\usepackage{subcaption}

% Better tables
\usepackage{longtable}
\usepackage{tabu}
\usepackage{booktabs}

% Mathematics
\usepackage{amsmath}

% Bold math symbols (e.g. vectors)
\usepackage{bm}

% Differential operators and derivatives
\usepackage{commath}

% Numbers and units
\usepackage{siunitx}
\sisetup{per-mode = symbol,
  inter-unit-product = \ensuremath{{}\cdot{}}}

% Misc LaTeX patches
\usepackage{fixltx2e}

% Compact list environments
\usepackage{mdwlist}

% Title without author and date
\makeatletter
\renewcommand{\@maketitle}{\newpage\null\begin{center}{\LARGE \@title \par}\end{center}\par}
\makeatother

% Description environment labels are not bold and are followed by colon
\renewcommand*\descriptionlabel[1]{\hspace\labelsep\normalfont #1:}

% Bold vectors
\renewcommand{\vec}[1]{\bm{#1}}
\newcommand{\uvec}[1]{\bm{\hat{#1}}}

% Del operator
\renewcommand{\del}{\vec{\nabla}}

% Volume
\newcommand{\volsym}{\settowidth{\dimen255}{V}\ooalign{$V$\cr\rule[.7ex]{\dimen255}{.4pt}\cr}}

% Error function and its complement
\DeclareMathOperator\erf{erf}
\DeclareMathOperator\erfc{erfc}

% Copyright notice on first page
\usepackage{fancyhdr}
\renewcommand{\headrulewidth}{0pt}

% Hyperlinks and PDF document title
\usepackage[pdftex]{hyperref}
\hypersetup{colorlinks,citecolor=black,filecolor=black,linkcolor=black,urlcolor=black}

\newcommand{\uuvec}[1]{\widehat{\vec{u}_{#1}}}
\usepackage{pict2e}

\begin{document}

% Copyright (C) 2013-2015  Jim Turner
%
% This work is licensed under the Creative Commons Attribution-ShareAlike 4.0 International License.
% To view a copy of this license, visit https://creativecommons.org/licenses/by-sa/4.0/.

\fancypagestyle{plain}{
\fancyfoot[C]{Copyright \textcopyright\ 2013--2015 Jim Turner. This work is licensed under the Creative
  Commons Attribution-ShareAlike~4.0 International License. To view a copy of this license, visit
  \mbox{\url{https://creativecommons.org/licenses/by-sa/4.0/}}. The \LaTeX{} source files are
  available under the same license at \mbox{\url{https://github.com/jturner314/engineering-equations}}.}
}

\maketitle
\tableofcontents

\section*{Thanks}

The references for the equations are listed in the References section.  Thanks also to those
students who used this equations sheet and provided feedback.  This feedback was very helpful for
finding errors in equations, discovering missing equations, etc.

\section*{Contributing}

This document and its sources are available under a free license (see the footer) in the hopes that
you will find them useful and so that you have the freedom to make them even better.  All
development for this equation sheet and others is done on GitHub.  If you'd like to contribute,
please visit the project page at the URL provided in the footer.  If you find an error or other area
for improvement, please create an issue or send a pull request via GitHub.

\newpage


\section{Particle Motion}
\label{sec:particle-motion}

\subsection{Kinematics}
\label{sec:particle-kinematics}

\begin{description*}
\item[Rectilinear motion]~
  \begin{description*}
    \item[Position] $s$
    \item[Velocity]
      \(v = \od{s}{t}\)
    \item[Acceleration]
      \(a = \od{v}{t}\)
    \item[Time-independent relationship]
      \(a\dif{s} = v\dif{v}\)
  \end{description*}
\item[Rectilinear motion with constant acceleration]~
  \begin{description*}
  \item[Velocity]
    \(v = v_0 + a_\mrm{c}t\)
  \item[Position]
    \(s = s_0 + v_0t + \frac12a_\mrm{c}t^2\)
  \item[Time-independent relationship]
    \(v^2 = v_0^2 + 2a_\mrm{c}(s-s_0)\)
  \end{description*}
\item[Normal and tangential coordinates]~
  \begin{description*}
  \item[Velocity]
    \(\vec{v} = v\uvec{s}\)
  \item[Acceleration]
    \(\vec{a} = \dot{v}\uvec{s} + v\od{\uvec{s}}{t}
    = \dot{v}\uvec{s} + \frac{v^2}{\rho}\uvec{n}\)
  \item[Radius of curvature of plane curve]
    \(\rho = \frac{{\left[1+{\left(\od{y}{x}\right)}^2\right]}^{3/2}}{\left|\od[2]{y}{x}\right|}\)
  \end{description*}
\item[Cylindrical coordinates]~
  \begin{description*}
  \item[Position]
    \(\vec{r} = r\uuvec{r} + z\uuvec{z}\)
  \item[Velocity]
    \(\vec{v} = \dot{r}\uuvec{r} + r\dot{\theta}\uuvec{\theta} + \dot{z}\uuvec{z}\)
  \item[Acceleration]
    \(\vec{a}
    = (\ddot{r}-r\dot\theta^2)\uuvec{r}
    + (r\ddot\theta+2\dot{r}\dot\theta)\uuvec\theta
    + \ddot{z}\uuvec{z}\)
  \item[Angle $\psi$ between $\uuvec{r}$ and $\uvec{s}$]
    \(\tan\psi = r\od{\theta}{r}\)
  \end{description*}
\end{description*}

\subsection{Kinetics and Static forces}
\label{sec:particle-kinetics}

\begin{description*}
\item[Newton's second law]
  \(\sum\vec{F} = m\vec{a}\)
\item[Gravitational force]
  \(F_G = G\frac{m_1m_2}{r^2}\) where
  \(G = \SI{6.67e-11}{\cubic\meter\per\kilogram\per\square\second}\)
\item[Spring force]
  \(F_s = ks\) where $s$ is the displacement
\end{description*}

\subsection{Work and Energy}
\label{sec:particle-energy}

\begin{description*}
\item[Kinetic energy]
  \(T = \frac12mv^2\)
\item[Work]
  \(U_{1\rightarrow2} = \int_{\vec{r}_1}^{\vec{r}_2}\vec{F}\cdot\dif\vec{r}
  = \int_{s_1}^{s_2}F\cos\theta\dif{s} = \int_{\theta_1}^{\theta_2}M\dif{\theta}\)
\item[Work--Kinetic Energy Principle]
  \(\sum U_{1\rightarrow2} = T_2 - T_1\)
\item[Power]
  \(P = \od{}{t}U\)
\item[Gravitational potential energy]
  \(V_\mrm{g} = mgh\)
\item[Elastic potential energy]
  \(V_\mrm{e} = \frac12ks^2\)
\item[When only conservative forces are present]
  \(E = V_1 + T_1 = V_2 + T_2\)

\end{description*}

\subsection{Impulse and Momentum}
\label{sec:particle-momentum}

\begin{description*}
\item[Linear momentum]
  \(\vec{L} = m\vec{v}\)
\item[Linear impulse]
  \(\vec{I}_{1\rightarrow2} = \int_{t_1}^{t_2}\vec{F}\dif{t} = \vec{L}_2 - \vec{L}_1\)
\item[Rotational momentum]
  \(\vec{H}_O = \vec{r} \times (m\vec{v})\)
\item[Rotational impulse]
  \(\int_{t_1}^{t_2}\vec{M}_O\dif{t} = {(\vec{H}_O)}_2 - {(\vec{H}_O)}_1\)
\end{description*}

\subsection{Central Impact Collisions}
\label{sec:particle-collisions}

Collisions consist of a deformation impulse \(\int P \dif{t}\) as the particles push against each
other and deform, and a restitution impulse \(\int R \dif{t}\) as the particles push away from each
other.

\begin{description*}
\item[Deformation]
  \(mv_1 - \int P \dif{t} = mv_2\)
\item[Restitution]
  \(mv_2 - \int R \dif{t} = mv_3\)
\item[Coefficient of restitution]
  \(e = \frac{\int R \dif{t}}{\int P \dif{t}} = \frac{{(v_B)}_2 - {(v_A)}_2}{{(v_A)}_1 - {(v_B)}_1}\)
\end{description*}

\section{Rigid Body Planar Motion}
\label{sec:rigid-body-motion}

We analyze a rigid body as a system of particles attached to each other such that they move as one
unit. In general, rigid body motion can be expressed as a combination of translation and
rotation. $A$ and $B$ are particles within a rigid body.

\subsection{Kinematics}
\label{sec:rigid-body-kinematics}

\subsubsection{Angular Position, Velocity, and Acceleration}

\begin{description*}
\item[Angular position] \(\vec\theta\) where the direction of the vector is the axis about which the
  angular position is taken
\item[Angular velocity]
  \(\vec\omega = \dot{\vec\theta}\)
\item[Angular acceleration]
  \(\vec\alpha = \dot{\vec\omega}\)
\end{description*}

\subsubsection{Fixed axis rotation with constant angular acceleration \(a_\mrm{c}\)}

\begin{description*}
\item[Angular velocity]
  \(\omega = \omega_0 + \alpha_\mrm{c}t\)
\item[Angular position]
  \(\theta = \theta_0 + \omega_0t + \frac12\alpha_\mrm{c}t^2\)
\item[Time-independent relationship]
  \(\omega^2 = \omega_0^2 + 2\alpha_\mrm{c}(\theta-\theta_0)\)
\end{description*}

\subsubsection{Relative Motion Analysis}

The subscript ${}_{B/A}$ means ``of $B$ relative to $A$''.

\begin{description*}
\item[Position]
  \(\vec{r}_B = \vec{r}_A + \vec{r}_{B/A}\)
\item[Velocity]
  \(\vec{v}_B = \vec{v}_A + \vec{v}_{B/A}\)
\item[Acceleration]
  \(\vec{a}_B = \vec{a}_A + \vec{a}_{B/A}\)
\end{description*}

\subsubsection{General Planar Motion}

\begin{description*}
\item[Non-rotating reference frame]~
  \begin{description*}
  \item[Velocity]
    \(\vec{v}_{B/A} = \vec{\omega}\times\vec{r}_{B/A}\)
  \item[Acceleration]
    \(\vec{a}_{B/A} = \vec\alpha\times\vec{r}_{B/A} - \omega^2\vec{r}_{B/A}\)
  \end{description*}
\item[Rotating reference frame] axes $xyz$ rotating with angular velocity $\vec\Omega$
  relative to fixed axes $XYZ$
  \begin{description*}
  \item[Velocity]
    \(\vec{v}_{B/A} = \vec\Omega\times\vec{r}_{B/A} + {(\vec{v}_{B/A})}_{xyz}\)
  \item[Acceleration]
    \(\vec{a}_{B/A} = \dot{\vec\Omega}\times\vec{r}_{B/A} + \vec\Omega\times(\vec\Omega\times\vec{r}_{B/A})
    + 2\vec\Omega\times{(\vec{v}_{B/A})}_{xyz} + {(\vec{a}_{B/A})}_{xyz}\)
  \end{description*}
\end{description*}

\subsubsection{Instantaneous Center of Zero Velocity}

A combined planar translation and rotation at any instant in time can be treated as a rotation about
a specially chosen point called the instantaneous instantaneous center of zero velocity
($\mathit{IC}$). This simplifies finding the velocities of points within the body. (Note that in
general, this method does not work for accelerations because the $\mathit{IC}$ may be accelerating.)
The location of the $\mathit{IC}$ can be found from two known velocities on the rigid body by
constructing lines as shown in Figure~\ref{fig:instantaneous-center}. Using the $\mathit{IC}$, the
velocity of a point $A$ on the rigid body can by found by: \[v_A = \omega r_{A/\mathit{IC}}\]

\begin{figure}
  \centering
  \setlength{\unitlength}{2mm}
  \begin{subfigure}[b]{34mm}
    \centering
    \begin{picture}(13,16)(0,-2)
      \thicklines
      \put(0,0){\circle*{0.5}}
      \put(0,8){\circle*{0.5}}
      \put(0,8){\vector(1,0){6}}
      \put(0,12){\circle*{0.5}}
      \put(0,12){\vector(1,0){9}}

      \thinlines
      \put(0,0){\line(0,1){12}}
      \put(0,0){\line(3,4){9}}

      \thinlines
      \put(0,7){\line(1,0){1}}
      \put(1,7){\line(0,1){1}}
      \put(0,11){\line(1,0){1}}
      \put(1,11){\line(0,1){1}}

      \put(1,-0.5){$\mathit{IC}$}
      \put(7,7.5){$\vec{v}_A$}
      \put(10,11.5){$\vec{v}_B$}
    \end{picture}
    \caption{For parallel $\vec{v}$}
  \end{subfigure}
  \quad
  \begin{subfigure}[b]{34mm}
    \centering
    \begin{picture}(13,16)(0,-2)
      \thicklines
      \put(0,0){\circle*{0.5}}
      \put(6,8){\circle*{0.5}}
      \put(6,8){\vector(4,-3){4.8}}
      \put(0,12){\circle*{0.5}}
      \put(0,12){\vector(1,0){6}}

      \thinlines
      \put(0,0){\line(0,1){12}}
      \put(0,0){\line(3,4){6}}

      \thinlines
      \put(5.4,7.2){\line(4,-3){0.8}}
      \put(6.2,6.6){\line(3,4){0.6}}
      \put(0,11){\line(1,0){1}}
      \put(1,11){\line(0,1){1}}

      \put(1,-0.5){$\mathit{IC}$}
      \put(11.6,3.8){$\vec{v}_A$}
      \put(7,11.5){$\vec{v}_B$}
    \end{picture}
    \caption{For nonparallel $\vec{v}$}
  \end{subfigure}
  \caption{Geometric construction to find the instantaneous center of
    velocity}\label{fig:instantaneous-center}
\end{figure}

\subsection{Kinetics}
\label{sec:rigid-body-planar-kinetics}

\subsubsection{Moment of Inertia and Center of Mass}

\begin{description*}
\item[Moment of inertia]
  \(I_G = \int_m r^2 \dif{m}\)
\item[Parallel axis theorem]
  \(I_A = I_G + md^2\)
\item[Radius of gyration]
  \(k = \sqrt{I_G/m}\)
\item[Center of mass]
  \(\bar{x} = \frac{\int x \dif{m}}{\int \dif{m}}\) and
  \(\bar{y} = \frac{\int y \dif{m}}{\int \dif{m}}\)
\end{description*}

\subsubsection{Equations of Motion}
\label{sec:rigid-body-equations-of-motion}

\begin{description*}
\item[Translation only (no rotation)]~
  \begin{description*}
  \item[Forces]
    \(\sum \vec{F}_i = m\vec{a}_G\)
  \item[Moments]
    \(\sum {(M_i)}_G = 0\) or \(\sum {(M_i)}_A = m a_G d\)
  \end{description*}
\item[Fixed axis rotation (no translation)]~
  \begin{description*}
  \item[Forces]
    \(\sum {(F_i)}_r = -m r_G \omega^2\) and
    \(\sum {(F_i)}_\theta = m r_G \alpha\)
  \item[Moments]
    \(\sum {(M_i)}_G = I_G \alpha\) or
    \(\sum {(M_i)}_O = I_O \alpha\)
  \end{description*}
\item[General 2-D motion]~
  \begin{description*}
  \item[Forces]
    \(\sum \vec{F}_i = m \vec{a}_G\)
  \item[Moments]
    \(\sum {(M_i)}_A = -\bar{y} m {(a_G)}_x + \bar{x} m {(a_G)}_y + I_G \alpha\)
  \end{description*}
\end{description*}

\subsection{Work and Energy}
\label{sec:rigid-body-energy}

\begin{description*}
\item[Kinetic energy]
  \(T = \frac12 m v_G^2 + \frac12 I_G \omega^2\)
\item[Other equations] Particle energy equations (Section~\ref{sec:particle-energy}) also apply.
\end{description*}

\subsection{Impulse and Momentum}
\label{sec:rigid-body-momentum}

\begin{description*}
\item[Linear momentum]
  \(\vec{L} = m\vec{v}_G\)
\item[Linear impulse]
  \(\vec{I}_{1\rightarrow2} = \vec{L}_2 - \vec{L}_1\)
\item[Rotational momentum]
  \(\vec{H}_G = I_G \vec{\omega}\) or
  \(H_A = I_G \omega \pm m v_G d\)
\item[Rotational impulse]
  \(\int_{t_1}^{t_2}\vec{M}_G\dif{t} = {(\vec{H}_G)}_2 - {(\vec{H}_G)}_1\)
\end{description*}

\begin{thebibliography}{99}
\bibitem{Hibbeler2010} Hibbeler, R. C., 2010, \emph{Engineering Mechanics: Dynamics}, 12th ed.,
  Pearson Prentice Hall, Upper Saddle River, NJ.
\bibitem{Wu2012} Wu, Fen, 2012, \emph{Engineering Dynamics}, MAE 208 course lectures, North Carolina
  State University, Raleigh, NC.
\end{thebibliography}

\end{document}
