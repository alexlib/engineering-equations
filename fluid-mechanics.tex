% Copyright (C) 2013  Jim Turner
%
% This work is licensed under the Creative Commons Attribution-ShareAlike 3.0 Unported License. To
% view a copy of this license, visit http://creativecommons.org/licenses/by-sa/3.0/deed.en_US.

\documentclass{article}
\title{Fluid Mechanics Equations}

% Copyright (C) 2013  Jim Turner
%
% This work is licensed under the Creative Commons Attribution-ShareAlike 3.0 Unported License. To
% view a copy of this license, visit http://creativecommons.org/licenses/by-sa/3.0/deed.en_US.

% Page margins
\usepackage[margin=1in]{geometry}

% Muticolumn
\usepackage{multicol}

% Subcaptions
\usepackage{subcaption}

% Better tables
\usepackage{longtable}
\usepackage{tabu}
\usepackage{booktabs}

% Mathematics
\usepackage{amsmath}

% Bold math symbols (e.g. vectors)
\usepackage{bm}

% Differential operators and derivatives
\usepackage{commath}

% Numbers and units
\usepackage{siunitx}
\sisetup{per-mode = symbol,
  inter-unit-product = \ensuremath{{}\cdot{}}}

% Misc LaTeX patches
\usepackage{fixltx2e}

% Compact list environments
\usepackage{mdwlist}

% Title without author and date
\makeatletter
\renewcommand{\@maketitle}{\newpage\null\begin{center}{\LARGE \@title \par}\end{center}\par}
\makeatother

% Description environment labels are not bold and are followed by colon
\renewcommand*\descriptionlabel[1]{\hspace\labelsep\normalfont #1:}

% Bold vectors
\renewcommand{\vec}[1]{\bm{#1}}
\newcommand{\uvec}[1]{\bm{\hat{#1}}}

% Del operator
\renewcommand{\del}{\vec{\nabla}}

% Volume
\newcommand{\volsym}{\settowidth{\dimen255}{V}\ooalign{$V$\cr\rule[.7ex]{\dimen255}{.4pt}\cr}}

% Error function and its complement
\DeclareMathOperator\erf{erf}
\DeclareMathOperator\erfc{erfc}

% Copyright notice on first page
\usepackage{fancyhdr}
\renewcommand{\headrulewidth}{0pt}

% Hyperlinks and PDF document title
\usepackage[pdftex]{hyperref}
\hypersetup{colorlinks,citecolor=black,filecolor=black,linkcolor=black,urlcolor=black}


\begin{document}

% Copyright (C) 2013-2015  Jim Turner
%
% This work is licensed under the Creative Commons Attribution-ShareAlike 4.0 International License.
% To view a copy of this license, visit https://creativecommons.org/licenses/by-sa/4.0/.

\fancypagestyle{plain}{
\fancyfoot[C]{Copyright \textcopyright\ 2013--2015 Jim Turner. This work is licensed under the Creative
  Commons Attribution-ShareAlike~4.0 International License. To view a copy of this license, visit
  \mbox{\url{https://creativecommons.org/licenses/by-sa/4.0/}}. The \LaTeX{} source files are
  available under the same license at \mbox{\url{https://github.com/jturner314/engineering-equations}}.}
}

\maketitle
\tableofcontents

\section*{Thanks}

The references for the equations are listed in the References section.  Thanks also to those
students who used this equations sheet and provided feedback.  This feedback was very helpful for
finding errors in equations, discovering missing equations, etc.

\section*{Contributing}

This document and its sources are available under a free license (see the footer) in the hopes that
you will find them useful and so that you have the freedom to make them even better.  All
development for this equation sheet and others is done on GitHub.  If you'd like to contribute,
please visit the project page at the URL provided in the footer.  If you find an error or other area
for improvement, please create an issue or send a pull request via GitHub.

\newpage


\section*{A Note on Notation}

Fluid mechanics has conflicting notation styles and parameters.  In this document, pressure is
written in some places as $P$ and in others as $p$. These are equivalent.  Volume and velocity are
both represented with the English letter V.  To distinguish between them, velocity is written with a
normal $V$ while volume is written with a struck-through $\volsym$.  Note also that the conventional
symbols from engineering mechanics are used for linear momentum, $\vec{L}$, and angular momentum,
$\vec{H}$.

\section{Important Fluid Properties}

\subsection{Density, etc.}
\begin{description*}
\item[Density of air at STP]
  \(\rho_\mathrm{air} = \SI{1.204}{\kilo\gram\per\cubic\meter}\)
\item[Density of water]
  \(\rho_\mathrm{H_2O} = \SI{1000}{\kilo\gram\per\cubic\meter}\)
\item[Specific volume]
  \(v = 1/\rho\)
\item[Specific weight]
  \(\gamma = \rho g\)
\item[Specific gravity]
  \(\mathit{SG} = \rho / \rho_\mathrm{H_2O}\)
\end{description*}

\subsection{Viscosity}
\begin{description*}
\item[Newtonian fluids]
  \(\tau = \mu \od{u}{y}\)
\item[Kinematic viscosity]
  \(\nu = \mu / \rho\)
\end{description*}

\subsection{Speed of Sound}
\begin{description*}
\item[Speed of sound in air]
  \(c_\mathrm{air} = \SI{330}{\meter\per\second}\)
\item[Speed of sound in water]
  \(c_\mathrm{H_2O} = \SI{1560}{\meter\per\second}\)
\item[Speed of sound]
  \(c = \sqrt{\left(\od{P}{\rho}\right)_s}\)
\item[Mach number]
  \(\mathit{Ma} = u/c\)
\end{description*}

\subsection{Surface Tension}
\begin{description*}
\item[Definition] Surface tension can be defined as surface energy per unit area
  (\si{\joule\per\square\meter}) or normal force per unit length cut into surface of fluid
  (\si{\newton\per\meter}): \(F = \sigma \dif L\).
\item[Laplace pressure] Pressure difference between inside and outside of a curved surface due to
  surface tension
  \begin{description*}
  \item[Pressure inside a fluid droplet]
    \(\Delta P = P_\mathrm{in} -
    P_\mathrm{out} = \frac{2\sigma}{R}\)
  \item[Pressure inside a bubble] \(\Delta P = P_\mathrm{in} -
    P_\mathrm{out} = \frac{4\sigma}{R}\)
  \end{description*}
\end{description*}

\subsection{Ideal Gases}
\begin{description*}
\item[Universal gas constant]
  \(R_\mathrm{u} = \SI{8.31}{\joule\per\mole\per\kelvin}\)
\item[Ideal gas law]
  \(P = \rho R T\) where \(R = R_\mathrm{u}/\bar{M}\)
\item[Specific heats]
  \(c_v = \pd{\hat u}{T}\) and \(c_P = \pd{\hat h}{T}\)
\item[Relationship between $c_v$ and $c_p$]
  \(c_P = c_v + R\)
\item[Specific heat ratio]
  \(k = c_P/c_v\)
\end{description*}

\section{Fluid Statics}

\subsection{Pressure Gradient}
\begin{description*}
\item[Force (per unit volume) on fluid due to pressure gradient]
  \(\vec{f}_\mathrm{press} = -\del p\)
\item[Pressure gradient due to accelerating reference frame]
  \(\del p = \rho \left(\vec{g}-\vec{a}\right)\)
\item[Pressure due to gravity]
  \(\del p = \rho \vec g \implies p = \rho g h\)
\end{description*}

\subsection{Hydrostatic Forces on Surfaces}
\begin{description*}
\item[Plane surface inclined by angle $\theta$] Equivalent force acts normal to the surface at its
  center of pressure (CP), calculated using the pressure at the center of area (CA).
  \begin{description*}
  \item[Equivalent force]
    \(F = p_\mathrm{CA}A\)
  \item[Distance of CP from CA along the surface]
    \(y_\mathrm{CP} = -\rho g \sin\theta\frac{I_{xc}}{p_\mathrm{CA}A}\) and
    \(x_\mathrm{CP} = -\rho g \sin\theta\frac{I_{xyc}}{p_\mathrm{CA}A}\)
  \end{description*}
\item[Curved surface] Consider horizontal and vertical components of the equivalent force
  separately. Each component is equivalent to the force that would act on a projection of the curved
  surface onto a plane normal to the component. As a result, the components may act at different
  locations.
\end{description*}

\subsection{Archimedes's Laws}
\begin{description*}
\item[1st Law] A body immersed in a fluid experiences a vertical buoyant force equal to the weight
  of the fluid it displaces: \(F_\mathrm{B} = -W_\mathrm{wb}\)
\item[2nd Law] A floating body displaces its own weight in the fluid in which it floats:
  \(W_\mathrm{b} = W_\mathrm{wb}\)
\end{description*}

\subsection{Rotating Liquid with Constant Angular Velocity}
\begin{description*}
\item[Within fluid]
  \(z = \frac{P-P_0}{\rho g} + \frac{r^2\omega^2}{2g}\)
\item[Overall paraboloid height]
  \(h = \frac{R^2\omega^2}{2g}\)
\item[Volume of paraboloid]
  \(\volsym_p = \frac{1}{2} \pi R^2 h\)
\item[Initial fluid height]
  \(x = \frac{1}{2} h\)
\end{description*}

\section{Basic Fluid Dynamics}

\subsection{Bernoulli Equation}
\label{sec:bernoulli-eq}
\begin{description*}
\item[Conditions] Steady, incompressible, frictionless flow along streamline
\item[Pressure form]
  \(\frac{1}{2}\rho V^2 + \rho g h + P = \text{constant}\)
\item[Static pressure]
  \(P\)
\item[Stagnation pressure]
  \(P_0 = P + \frac{1}{2}\rho V^2\)
\item[Dynamic pressure]
  \(\frac{1}{2}\rho V^2\)
\item[Pitot tube]
  \(V = \sqrt{\frac{2\left(P_0-P_1\right)}{\rho}}\)
\item[Total head]
  \(H = \frac{V^2}{2g} + \frac{P}{\rho g} + h\)
\item[Turbine\slash{}pump head]
  \(h = \frac{\dot{W}}{\dot{m} g} = \frac{w}{g}\)
\item[Energy grade line] total head available to the fluid
\item[Hydraulic grade line] sum of pressure head and elevation head (does not include velocity head)
\end{description*}

\subsection{Velocity}
\begin{description*}
\item[Velocity field]
  \(\vec{V} = \od{\vec{r}}{t} = \vec{V}(x,y,z,t)
  = u\uvec{\mathrm{\i}}+v\uvec{\mathrm{\j}} + w\uvec{\mathrm{k}}\)
\item[Lines of interest]~
  \begin{description*}
  \item[Streamline] a line that is tangent to the velocity field along its full length
  \item[Pathline] the path that a fluid particle travels over time
  \item[Streakline] a line formed by particles that have previously passed through a particular point
  \end{description*}
\item[Finding streamline equation]
  \(\frac{\dif r}{V} = \frac{\dif x}{u} = \frac{\dif y}{v} = \frac{\dif z}{w}\)
\end{description*}

\subsection{Material Derivative Operator}
\begin{description*}
\item[Material derivative operator]
  \(\frac{\Dif{(\;)}}{\Dif t} = \pd{(\;)}{t} + (\vec V \cdot \del)(\;)
  = \pd{(\;)}{t} + u\pd{(\;)}{x} + v\pd{(\;)}{y} + w\pd{(\;)}{z}\)
\end{description*}

\subsection{Acceleration}
\begin{description*}
\item[Acceleration field]
  \(\vec a = \frac{\Dif\vec{V}}{\Dif t} = \pd{\vec V}{t} + (\vec{V}\cdot\del)\vec{V}\)
\item[$x$ component]
  \(a_x = \pd{u}{t} + u\pd{u}{x} + v\pd{u}{y} + w\pd{u}{z}\)
\item[Streamwise and normal components]
  \(a_s = V\pd{V}{s}\) and \(a_n = \frac{V^2}{R}\)
  where $R$ is the radius of curvature
\end{description*}

\section{Reynolds Transport Theorem (RTT)}
\begin{description*}
\item[Property per unit mass]
  \(\beta = B/m\)
\item[RTT, general form]
  \(\frac{\Dif B_\mathrm{sys}}{\Dif t}
  = \pd{B_\mathrm{cv}}{t}+\dot{B}_\mathrm{out} - \dot{B}_\mathrm{in}
  = \pd{}{t}\left(\int_\mathrm{cv}\beta\rho\dif{\volsym}\right)
  + \int_\mathrm{cs}\beta\rho(\vec{V}\cdot\uvec{n})\dif{A}\)
  where
  \(B_\mathrm{sys} = \int_\mathrm{sys}\beta\rho\dif{\volsym}\)
\end{description*}

\subsection{Mass}
\begin{description*}
\item[RTT, $M$]
  \(0 = \frac{\Dif M_\mathrm{sys}}{\Dif t}
  = \pd{M_\mathrm{cv}}{t}+\dot{m}_\mathrm{out}-\dot{m}_\mathrm{in}
  = \pd{}{t}\left(\int_\mathrm{cv}\rho\dif{\volsym}\right)
  + \int_\mathrm{cs}\rho(\vec{V}\cdot\uvec{n})\dif{A}\)
\item[Steady flow]
  \(\sum\dot{m}_\mathrm{in} = \sum\dot{m}_\mathrm{out}\)
\item[Mass flow rate for 1-D flow]
  \(\dot{m} = \rho A \bar{V} = \rho Q\)
\end{description*}

\subsection{Linear Momentum}
\begin{description*}
\item[RTT, $\vec{L}$]
  \(\sum{\vec{F}}
  = \frac{\Dif (m\vec{V})_\mathrm{sys}}{\Dif t}
  = \pd{(m\vec{V})_\mathrm{cv}}{t}
  + \sum\left(\dot{m}_i\vec{V}_i\right)_\mathrm{out}
  - \sum\left(\dot{m}_i\vec{V}_i\right)_\mathrm{in}
  = \pd{}{t}\left(\int_\mathrm{cv}\rho\vec{V}\dif{\volsym}\right)
  + \int_\mathrm{cs}\rho\vec{V}(\vec{V}\cdot\uvec{n})\dif{A}\)
\item[Steady flow]
  \(\sum{\vec{F}}
  = \sum\left(\dot{m}_i\vec{V}_i\right)_\mathrm{out}
  - \sum\left(\dot{m}_i\vec{V}_i\right)_\mathrm{in}\)
\item[Possible forces]~
  \begin{description*}
    \item[Pressure force]
      \(\sum\vec{F}_\mathrm{press} = \int_\mathrm{cs}p_\mathrm{gage}(-\uvec{n})\dif{A}\)
    \item[Gravity force] Use weight of fluid
    \item[Other body forces] Depend on each situation
  \end{description*}
\item[Moving control volume] For a control volume moving at constant velocity, use the same
  equations but with all velocities relative to the control volume.
\end{description*}

\subsection{Angular Momentum}
\begin{description*}
\item[RTT, $\vec{H}$]
  \(\sum\vec{M} = \sum\vec{r}\times\vec{F}
  = \frac{\Dif \vec{H}_\mathrm{sys}}{\Dif t}
  = \pd{}{t}\left(\int_\mathrm{cv}\rho(\vec{r}\times\vec{V})\dif{\volsym}\right)
  + \int_\mathrm{cs}(\vec{r}\times\vec{V})\rho(\vec{V}\cdot\uvec{n})\dif{A}\)
\item[1-D uniform flow inlets and outlets]
  \(\int_\mathrm{cs}(\vec{r}\times\vec{V})\rho(\vec{V}\cdot\uvec{n})\dif{A}
  = \sum(\dot{m}_i(\vec{r}\times\vec{V})_i)_\mathrm{out}
  - \sum(\dot{m}_i(\vec{r}\times\vec{V})_i)_\mathrm{in}\)
\end{description*}

\subsection{Energy}
\begin{description*}
\item[RTT, $E$]
  \(\frac{\Dif E_\mathrm{sys}}{\Dif t}
  = \od{Q}{t}-\od{W}{t}
  = \pd{}{t}\left(\int_\mathrm{cv}e\rho\dif{\volsym}\right)
  + \int_\mathrm{cs}e\rho(\vec{V}\cdot\uvec{n})\dif{A}\)
\item[RTT using enthalpy]
  \(\dot{Q}-\dot{W}_\mathrm{visc}-\dot{W}_\mathrm{shaft}
  = \pd{}{t}\left(\int_\mathrm{cv}e\rho\dif{\volsym}\right)
  + \int_\mathrm{cs}\rho(\hat{h}+\frac{1}{2}V^2+gz)(\vec{V}\cdot\uvec{n})\dif{A}\)
\item[1-D uniform flow inlets and outlets]~\\
  \(\int_\mathrm{cs}\rho(\hat{h}+\frac{1}{2}V^2+gz)(\vec{V}\cdot\uvec{n})\dif{A}
  = \sum\left(\dot{m}_i(\hat{h}+\frac{1}{2}V^2+gz)_i\right)_\mathrm{out}
  - \sum\left(\dot{m}_i(\hat{h}+\frac{1}{2}V^2+gz)_i\right)_\mathrm{in}\)
\item[Work of viscous forces]
  \(\dot{W}_\mathrm{visc} = -\int_\mathrm{cs}\vec\tau\cdot\vec{V}\dif{A}\)
\item[Extended Bernoulli Equation (EBE)]~
  \begin{description*}
  \item[Energy formulation]
    \(q - w_\mathrm{visc} - w_\mathrm{shaft}
    = \left(\hat{u}+\frac{P}{\rho}+\frac{1}{2}V^2+gz\right)_\mathrm{out}
    - \left(\hat{u}+\frac{P}{\rho}+\frac{1}{2}V^2+gz\right)_\mathrm{in}\)
  \item[Alternatively]
    Let \(e_\mathrm{loss} = \hat{u}_\mathrm{out}-\hat{u}_\mathrm{in}-q+w_\mathrm{visc}\), then \\
    \(\left(\frac{P}{\rho}+\frac{1}{2}V^2+gz\right)_\mathrm{out}
    = \left(\frac{P}{\rho}+\frac{1}{2}V^2+gz\right)_\mathrm{in}
    - w_\mathrm{shaft} - e_\mathrm{loss}\)
  \item[Head formulation]
    \(H_\mathrm{out} = H_\mathrm{in} + (h_P-h_T) - h_L\)
    where total head $H$ and turbine\slash{}pump head $h$ are defined in
    Section \ref{sec:bernoulli-eq}
  \end{description*}
\end{description*}

\section{Differential Analysis of Fluid Flow}

\subsection{Mass Conservation (Continuity Equation)}
\begin{description*}
\item[Continuity equation]
  \(0 = \pd{\rho}{t} + \del\cdot\rho\vec{V}
  = \pd{\rho}{t}+\pd{\rho u}{x}+\pd{\rho v}{y}+\pd{\rho w}{z}\)
\item[Cylindrical coordinates]
  \(0 = \pd{\rho}{t} + \frac{1}{r}\pd{(r\rho v_r)}{r}
  + \frac{1}{r}\pd{(\rho v_\theta)}{\theta} + \pd{(\rho v_z)}{z}\)
\item[Steady flow]
  \(0 = \del\cdot\rho\vec{V}\)
\item[Stream function for 2-D flow ($\psi$ is constant along streamlines)]~
  \begin{description*}
  \item[Rectangular coordinates]
    \(u = \pd{\psi}{y}\) and \(v = -\pd{\psi}{x}\)
  \item[Cylindrical coordinates]
    \(v_r = \frac{1}{r}\pd{\psi}{\theta}\) and \(v_\theta = -\pd{\psi}{r}\)
  \item[Volume flow rate between streamlines]
    \(q = \psi_2 - \psi_1\)
  \end{description*}
\end{description*}

\subsection{Fluid Rotation}
\begin{description*}
\item[Vorticity vector]
  \(\vec\varsigma = \del\times\vec{V}\)
\item[Irrotational flow]
  \(\del\times\vec{V} = 0\)
\item[Velocity potential $\phi(x,y,z,t)$] for irrotational, inviscid flow
  \begin{description*}
  \item[Definition]
    \(\vec V = \del\phi\)
  \item[Rectangular coordinates]
    \(u = \pd{\phi}{x}\),
    \(v = \pd{\phi}{y}\),
    \(w = \pd{\phi}{z}\)
  \item[Cylindrical coordinates]
    \(v_r = \pd{\phi}{r}\),
    \(v_\theta = \frac{1}{r}\pd{\phi}{\theta}\),
    \(v_z = \pd{\phi}{z}\)
  \item[Laplace's equation] due to irrotationality,
    \(\del^2\psi = 0\)
  \end{description*}
\end{description*}

\subsection{Navier-Stokes Equations}
\begin{description*}
\item[Assumptions] While the Navier-Stokes equations can apply more generally, the equations in this
  sheet assume Newtonian, incompressible fluid flow
\item[Navier-Stokes equations]
  \(\rho\vec{g} - \del p + \mu\del^2\vec{V} = \rho\frac{\Dif\vec V}{\Dif t}\)
  \begin{description*}
  \item[Rectangular, $x$ direction]
    \(\rho g_x - \pd{p}{x} + \mu\left(\pd[2]{u}{x}+\pd[2]{u}{y}+\pd[2]{u}{z}\right)
    = \rho\frac{\Dif u}{\Dif t} = \rho a_x\)
  \item[Cylindrical, $z$ direction]
    \(\rho\left(\pd{v_z}{t}+v_r\pd{v_z}{r}+\frac{v_\theta}{r}\pd{v_z}{\theta}+v_z\pd{v_z}{z}\right)
    = \rho g_z - \pd{p}{z}
    + \mu\left[\frac{1}{r}\pd{}{r}\left(r\pd{v_z}{r}\right)
      + \frac{1}{r^2}\pd[2]{v_z}{\theta}
      + \pd[2]{v_z}{z}\right]\)
  \end{description*}
\item[Euler equations] Equivalent to Navier-Stokes equations with $\mu=0$
\item[Couette flow] Steady, viscous flow between two parallel fixed plates caused pressure gradient
  along the length of the plates. Produces a parabolic velocity distribution between the plates.
  \begin{description*}
  \item[Definitions] $h$ is one half the distance between the plates;
    $y\in[-h,h]$ is the vertical coordinate from the midline;
    $b$ is the width of the plates
  \item[Velocity at position $y$]
    \(u = u_\mathrm{max}\left(1-\frac{y^2}{h^2}\right)\)
  \item[Maximum velocity]
    \(u_\mathrm{max} = \left(-\pd{P}{x}\right)\frac{h^2}{2\mu}\)
  \item[Average velocity]
    \(u_\mathrm{ave} = \frac{2}{3}u_\mathrm{max}\)
  \item[Flow rate]
    \(Q = \frac{4}{3}u_\mathrm{max}hb = \left(-\pd{P}{x}\right)\frac{2bh^3}{3\mu}\)
  \end{description*}
\item[Laminar flow in pipe] Steady, viscous flow within a cylindrical pipe caused by pressure
  gradient along the length of the pipe.
  \begin{description*}
  \item[Velocity at position $r$]
    \(v_z = v_\mathrm{max}\left(1 - \frac{r^2}{R^2}\right)\)
  \item[Maximum velocity]
    \(v_\mathrm{max} = \left(-\pd{P}{z}\right)\frac{R^2}{4\mu}\)
  \item[Average velocity]
    \(v_\mathrm{ave} = \frac{1}{2}v_\mathrm{max}\)
  \item[Flow rate]
    \(Q = \frac{1}{2}v_\mathrm{max}\pi R^2 = \left(-\pd{P}{z}\right)\frac{\pi R^4}{8\mu}\)
  \end{description*}
\end{description*}

\section{Dimensional Analysis}
\begin{description*}
\item[Pi theorem procedure]~
  \begin{enumerate*}
  \item List all $n$ independent dimensional parameters (e.g. $\rho$, $D$, $V$, $\mu$).
  \item List the $r$ basic dimensions of the parameters (e.g. mass, length, time, temperature).
  \item Write the parameters in terms of their primary dimensions (e.g. $\rho\rightarrow M/L^3$).
    See Table \ref{tab:dimensional} for some common parameters.
  \item Select $r$ dimensional parameters that include all of the basic dimensions
    (e.g. $\rho$, $V$, $D$).
  \item Write $n-r$ equations for the dimensionless $\Pi$ groups, with each group containing the
    selected parameters and one other (e.g. \(\Pi_1 = \rho^a V^b D^c \mu = M^0L^0t^0\)). Solve these
    equations to find the exponents (e.g. $a$, $b$, $c$).
  \item Write the dimensionless $\Pi$ groups in terms of the dimensional parameters and double check
    that they are dimensionless (e.g. \(\Pi_1 = \frac{\rho{}VD}{\mu}
    = \frac{\frac{M}{L^3}\frac{L}{t}L}{\frac{M}{Lt}} = \textrm{dimensionless}\)).
  \item Write one $\Pi$ group as a function of the other $\Pi$ groups
    (e.g. \(\Pi_1 = G(\Pi_1,\Pi_2,\Pi_3)\)).
  \end{enumerate*}
\item[Important parameters] See Table \ref{tab:params}
\end{description*}

\begin{table}
  \centering
  \subcaptionbox{Dimensional\label{tab:dimensional}}{
  \begin{tabu}{lc}
  \toprule
  Name & Dimensions \\
  \midrule
  mass, \(m\) & \(M\) \\
  length, \(\ell\) & \(L\) \\
  temperature, \(T\) & \(T\) \\
  time, \(t\) & \(t\) \\
  velocity, \(V\) & \(L/t\) \\
  force, \(F\) & \(ML/t^2\) \\
  energy, \(E\) & \(ML^2/t^2\) \\
  pressure, \(p\) & \(M/(Lt^2)\) \\
  volume, \(\volsym\) & \(L^3\) \\
  volume flow rate, \(Q\) & \(L^3/t\) \\
  density, \(\rho\) & \(M/L^3\) \\
  specific weight, \(\gamma\) & \(M/(L^2t^2)\) \\
  dynamic viscosity, \(\mu\) & \(M/(Lt)\) \\
  kinematic viscosity, \(\nu\) & \(L^2/t\) \\
  \bottomrule
  \end{tabu}}
  ~
  \subcaptionbox{Dimensionless\label{tab:dimensionless}}{
  \tabulinesep=_2pt^2pt
  \begin{tabu}{lcc}
  \toprule
  Name & Meaning & Definition \\
  \midrule
  Reynolds number, \(\mathit{Re}\) & \(\frac{\textrm{inertia force}}{\textrm{viscous force}}\)
  & \(\frac{\rho{}V\ell}{\mu}\) \\
  Euler number, \(\mathit{Eu}\) & \(\frac{\textrm{pressure force}}{\textrm{inertia force}}\)
  & \(\frac{\Delta p}{\rho{}V^2}\) \\
  Froude number, \(\mathit{Fr}\) & \(\frac{\textrm{inertia force}}{\textrm{gravity force}}\)
  & \(\frac{V}{\sqrt{g\ell}}\) \\
  Mach number, \(\mathit{Ma}\) & \(\frac{\textrm{inertia force}}{\textrm{compressibility force}}\)
  & \(\frac{V}{c}\) \\
  Cauchy number, \(\mathit{Ca}\) & \(\frac{\textrm{inertia force}}{\textrm{compressibility force}}\)
  & \(\frac{\rho{}V^2}{E_v}\) \\
  Weber number, \(\mathit{We}\) & \(\frac{\textrm{inertia force}}{\textrm{surface tension force}}\)
  & \(\frac{\rho{}V^2\ell}{\sigma}\) \\
  Strouhal number, \(\mathit{St}\)
  & \(\frac{\textrm{local inertia force}}{\textrm{convective inertia force}}\)
  & \(\frac{\omega\ell}{V}\) \\
  lift coefficient, \(C_\mathrm{L}\) & \(\frac{\textrm{lift force}}{\textrm{dynamic pressure force}}\)
  & \(\frac{F_\mathrm{L}}{\frac12\rho{}V^2\ell^2}\) \\
  drag coefficient, \(C_\mathrm{D}\) & \(\frac{\textrm{drag force}}{\textrm{dynamic pressure force}}\)
  & \(\frac{F_\mathrm{D}}{\frac12\rho{}V^2\ell^2}\) \\
  \bottomrule
  \end{tabu}}
  \caption{Important parameters}\label{tab:params}
\end{table}

\section{Viscous Pipe Flow}

\begin{description*}
\item[Major losses] head losses due to viscous effects in straight pipe sections; given by
  \(h_{L\mathrm{major}} = f\frac{L}{D}\frac{V^2}{2g}\)
\item[EBE with major losses only]
  \(p_1-p_2 = \rho{}g(z_2-z_1)+\rho{}gh_L = \rho{}g(z_2-z_1) + f\frac{L}{D}\frac{\rho{}V^2}{2}\)
\item[Absolute pipe roughness, $\varepsilon$] a measure of the roughness of pipe walls; has units of
  length. Search online or see Table 8.1 of~\cite{fofm} for values of $\varepsilon$.
\item[Friction factor, $f$] determine flow type with Reynolds number,
  \(\mathit{Re} = \frac{\rho{}VD}{\mu}\), then use:
  \begin{description}
  \item[Laminar (\(\mathit{Re}<2100\))]
    \(f = \frac{64}{\mathit{Re}}\)
  \item[Transitional (\(2100<\mathit{Re}<4000\))]
    highly variable $f$; in general, use turbulent value
  \item[Turbulent (\(\mathit{Re}>4000\))]
    \(\frac{1}{\sqrt{f}}
    = -1.8\log\left[\left(\frac{\varepsilon/D}{3.7}\right)^{1.11}
      + \frac{6.9}{\mathit{Re}}\right]\)
  \end{description}
  Alternatively, you can look up values graphically with the Moody diagram in Figure
  \ref{fig:moody}.
\end{description*}

\begin{figure}
  \centering
  \includegraphics{moody-diagram.eps}
  \caption{Moody diagram}
  \label{fig:moody}
\end{figure}

\begin{thebibliography}{99}
\bibitem{fofm} Munson, B. R., Okiishi, T. H., Huebsch, W. W., and Rothmayer, A. P., 2013,
  \emph{Fundamentals of Fluid Mechanics}, 7th ed., John Wiley \& Sons, Inc., Jefferson City.
\bibitem{slec} Saveliev, A. V., 2013, \emph{Fluid Mechanics}, MAE 308 course lectures, North
  Carolina State University, Raleigh, NC.
\end{thebibliography}

\end{document}
