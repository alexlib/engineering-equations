% Copyright (C) 2013  Jim Turner
%
% This work is licensed under the Creative Commons Attribution-ShareAlike 3.0 Unported License. To
% view a copy of this license, visit http://creativecommons.org/licenses/by-sa/3.0/deed.en_US.

\documentclass{article}
\title{Engineering Design Optimization}

% Copyright (C) 2013  Jim Turner
%
% This work is licensed under the Creative Commons Attribution-ShareAlike 3.0 Unported License. To
% view a copy of this license, visit http://creativecommons.org/licenses/by-sa/3.0/deed.en_US.

% Page margins
\usepackage[margin=1in]{geometry}

% Muticolumn
\usepackage{multicol}

% Subcaptions
\usepackage{subcaption}

% Better tables
\usepackage{longtable}
\usepackage{tabu}
\usepackage{booktabs}

% Mathematics
\usepackage{amsmath}

% Bold math symbols (e.g. vectors)
\usepackage{bm}

% Differential operators and derivatives
\usepackage{commath}

% Numbers and units
\usepackage{siunitx}
\sisetup{per-mode = symbol,
  inter-unit-product = \ensuremath{{}\cdot{}}}

% Misc LaTeX patches
\usepackage{fixltx2e}

% Compact list environments
\usepackage{mdwlist}

% Title without author and date
\makeatletter
\renewcommand{\@maketitle}{\newpage\null\begin{center}{\LARGE \@title \par}\end{center}\par}
\makeatother

% Description environment labels are not bold and are followed by colon
\renewcommand*\descriptionlabel[1]{\hspace\labelsep\normalfont #1:}

% Bold vectors
\renewcommand{\vec}[1]{\bm{#1}}
\newcommand{\uvec}[1]{\bm{\hat{#1}}}

% Del operator
\renewcommand{\del}{\vec{\nabla}}

% Volume
\newcommand{\volsym}{\settowidth{\dimen255}{V}\ooalign{$V$\cr\rule[.7ex]{\dimen255}{.4pt}\cr}}

% Error function and its complement
\DeclareMathOperator\erf{erf}
\DeclareMathOperator\erfc{erfc}

% Copyright notice on first page
\usepackage{fancyhdr}
\renewcommand{\headrulewidth}{0pt}

% Hyperlinks and PDF document title
\usepackage[pdftex]{hyperref}
\hypersetup{colorlinks,citecolor=black,filecolor=black,linkcolor=black,urlcolor=black}


\begin{document}

% Copyright (C) 2013-2015  Jim Turner
%
% This work is licensed under the Creative Commons Attribution-ShareAlike 4.0 International License.
% To view a copy of this license, visit https://creativecommons.org/licenses/by-sa/4.0/.

\fancypagestyle{plain}{
\fancyfoot[C]{Copyright \textcopyright\ 2013--2015 Jim Turner. This work is licensed under the Creative
  Commons Attribution-ShareAlike~4.0 International License. To view a copy of this license, visit
  \mbox{\url{https://creativecommons.org/licenses/by-sa/4.0/}}. The \LaTeX{} source files are
  available under the same license at \mbox{\url{https://github.com/jturner314/engineering-equations}}.}
}

\maketitle
\tableofcontents

\section*{Thanks}

The references for the equations are listed in the References section.  Thanks also to those
students who used this equations sheet and provided feedback.  This feedback was very helpful for
finding errors in equations, discovering missing equations, etc.

\section*{Contributing}

This document and its sources are available under a free license (see the footer) in the hopes that
you will find them useful and so that you have the freedom to make them even better.  All
development for this equation sheet and others is done on GitHub.  If you'd like to contribute,
please visit the project page at the URL provided in the footer.  If you find an error or other area
for improvement, please create an issue or send a pull request via GitHub.

\newpage


\section{Standard Form}\label{sec:standard-form}

Find \(\vec{x} =
\begin{Bmatrix}
  x_1 \\
  x_2 \\
  \vdots \\
  x_n
\end{Bmatrix}\)
which minimizes \(f(\vec{x})\) subject to the constraints
\(\begin{aligned}
  g_j(\vec{x}) &\le 0, \ j = 1, 2, \dots, m \\
  h_k(\vec{x}) &\le 0, \ k = 1, 2, \dots, l
\end{aligned}\)

\section{Conditions for Optimality}\label{sec:conditions-optimality}

\subsection{Unconstrained}\label{sec:unconstrained-conditions}

\begin{description*}
  \item[Necessary conditions]~
    \begin{itemize*}
      \item \(\del F(x) = 0\)
      \item \(H(x)\) is positive semi-definite or positive definite
    \end{itemize*}
  \item[Sufficient conditions]~
    \begin{itemize*}
      \item \(H(x)\) is positive definite
    \end{itemize*}
\end{description*}

\subsection{Constrained}\label{sec:constrained-conditions}

Kuhn--Tucker Conditions are necessary for a relative minimum at \(\vec{x}^*\) and are sufficient to
ensure a global minimum at \(\vec{x}^*\) for convex programming problems:
\begin{enumerate*}
  \item \(\vec{x}^*\) is feasible (meets constraints)
  \item \(\lambda_j g_j(\vec{x}^*) = 0\) where \(j = 1, 2, \dots, m\) and \(\lambda_j \ge 0\)
  \item \(0 = \del f(\vec{x}^*) + \sum_{j=1}^m{\lambda_j \del g_j(\vec{x}^*)} +
    \sum_{k=1}^l{\lambda_{m+k} \del h_k(\vec{x}^*)}\)
    where \(\lambda_{m+k}\) are unrestricted in sign
\end{enumerate*}
To solve for \(\lambda_j\) for problems with only inequality constraints, let
\[\vec{B} = \del f(\vec{x}^*) \textrm{ and } A =
\begin{bmatrix}
  \del g_1(\vec{x}^*) & \del g_2(\vec{x}^*) & \cdots & \del g_n(\vec{x}^*)
\end{bmatrix}\]
where \(A\) only contains active constraints, then
\[\vec{\lambda} = -{\left[A^\top A\right]}^{-1} A^\top \vec{B}\]

\section{1-D Optimization}\label{sec:1-d-optimization}

\subsection{Bounding\slash{}Range Finding}\label{sec:bounding}

\begin{enumerate*}
\item \label{itm:choose-init-and-step} Choose \(x^0\) and a step size \(\Delta\). Set \(k = 0\). The
  larger the value of \(\Delta\), the fewer the function calls required for bounding but the larger
  the final bounds will be.
\item If \(f(x^0 - \abs{\Delta}) \ge f(x^0) \ge f(x^0 + \abs{\Delta})\),
  then \(\Delta\) is positive \\
  else if \(f(x^0 - \abs{\Delta}) \le f(x^0) \le f(x^0 + \abs{\Delta})\),
  then \(\Delta\) is negative \\
  else go to step~\ref{itm:choose-init-and-step}.
\item \label{itm:calc-next-x} \(x^{k+1} = x^k + 2^k \Delta\)
\item If \(f(x^{k+1}) < f(x^k)\),
  then \(k := k + 1\) and go to step~\ref{itm:calc-next-x} \\
  else the minimum is between \((x^{k-1}, x^{k+1})\)
\end{enumerate*}

\subsection{Interpolation\slash{}Approximation Methods}\label{sec:interp-approx}

Interpolation\slash{}approximation methods approximate the objective function with simpler functions
for which the minima are known, then iteratively improve their approximations. They are more
unpredictable and less robust than region elimination methods but converge faster for continuous and
well-conditioned problems.

\subsubsection{Three-Point Quadratic Non-Derivative Optimization with Refinement}\label{sec:quadratic-interpolation}

\begin{enumerate*}
\item Start with three points: \((x_1, f_1)\), \((x_2, f_2)\), \((x_3, f_3)\) where \(x_1\),
  \(x_3\) bound the minimum and \(x_2 \in (x_1, x_3)\). Typically, \(x_2 = \frac12(x_1 + x_3)\).
\item \label{itm:define-interp} Define an interpolation function \(\tilde{f}(x) = a_0 + a_1(x - x_1)
  + a_2(x - x_1)(x - x_2) \approx f(x)\)
\item Find \(f_\mrm{min} = \min{(f_1, f_2, f_3)}\) and associated \(x_\mrm{min}\).
\item To find the interpolation function, \(\tilde{f}(x)\):
\begin{align*}
  a_0 &= f_1 \\
  a_1 &= \frac{f_2 - f_1}{x_2 - x_1} \\
  a_2 &= \frac{1}{x_3 - x_2}\left(\frac{f_3 - f_1}{x_3 - x_1} - \frac{f_2 - f_1}{x_2 - x_1}\right)
\end{align*}
Then at the minimum of \(\tilde{f}(x)\), \(\left.\pd{\tilde{f}}{x}\right|_{\tilde{x}^*} = a_1 +
a_2(\tilde{x}^* - x_2) + a_2(\tilde{x}^* - x_1) = 0\). Solving for \(\tilde{x}^*\):
\begin{align*}
\tilde{x}^* &= \frac{x_2 + x_1}{2} - \frac{a_1}{2a_2} \\
\tilde{f}^* &= f(\tilde{x}^*)
\end{align*}
\item Check convergence and if converged, stop:
\[\abs{\frac{\tilde{x}^* - x_\mrm{min}}{x_\mrm{min}}} \le \epsilon_x \qquad \qquad
\abs{\frac{\tilde{f}^* - f_\mrm{min}}{f_\mrm{min}}} \le \epsilon_f\]
\item Save the best point (either \(\tilde{x}^*\) or \(x_\mrm{min}\)) and the two points that
  bracket it. Relabel the saved points as \(x_1\), \(x_2\), \(x_3\). Recalculate \(f_1\), \(f_2\),
  \(f_3\). Go to step~\ref{itm:define-interp}.
\end{enumerate*}

\subsubsection{Newton--Raphson Method}\label{sec:newton-raphson}

This method uses first and second derivative information to speed up convergence. However,
discontinuities are a problem, for some problems it can be expensive to get the derivative, and the
algorithm can diverge in some cases.
\[\tilde{f}'(x) = f'(x^k) + f''(x^k)(x - x^k)\]
Solve \(\tilde{f}'(x^{k+1}) = 0\) for \(x^{k+1}\):
\[x^{k+1} = x^k - \frac{f'(x^k)}{f''(x^k)}\]
Iterate until \(\abs{x^{k+1} - x^k}\) is small. Note that divergence is possible.

\subsubsection{Two-Point Cubic Optimization with Refinement}\label{sec:cubic-interpolation}

This interpolation-based method uses first derivative information to generate splines for
\(\tilde{f}(x)\) and \(\tilde{f}'(x)\).

\subsection{Region Elimination Methods}\label{sec:region-elimination}

Region elimination methods iteratively eliminate subintervals of the design space from
consideration. They generally eliminate the same percentage of the space on each iteration. They are
more robust than interpolation\slash{}approximation method, particularly for discontinuous or
ill-conditioned functions, but may be slower for well-conditioned problems.

\subsubsection{Interval Halving Method}\label{sec:interval-halving}

After \(n\) calls to \(f\), the space is reduced to about \({\left(\frac12\right)}^{n/2}\) of its
original size.

\begin{enumerate*}
\setcounter{enumi}{-1}
\item Should already know \((x_\mrm{L}, f_\mrm{L})\) and \((x_\mrm{R}, f_\mrm{R})\) from bounding
  algorithm.
\item Calculate the following:
  \begin{align}
    x_\mrm{m} &= \frac{x_\mrm{L} + x_\mrm{R}}{2} \nonumber \\
    f_\mrm{m} &= f(x_\mrm{m}) \nonumber \\
    L &= x_\mrm{R} - x_\mrm{L} \label{eq:interval-halving-L}
  \end{align}
\item \label{itm:update-x1-x2} Calculate the following:
  \begin{align*}
    x_1 &= x_\mrm{L} + \frac{L}{4} \\
    f_1 &= f(x_1) \\
    x_2 &= x_\mrm{R} - \frac{L}{4} \\
    f_2 &= f(x_2)
  \end{align*}
\item Set \(x_\mrm{L}\), \(x_\mrm{m}\), and \(x_\mrm{R}\) to the three points that make a V shape.
\item Compute \(L\) using equation~\ref{eq:interval-halving-L} and check convergence. Go to
  step~\ref{itm:update-x1-x2} if not converged.
\end{enumerate*}

\subsubsection{Golden Section Method}\label{sec:golden-section-region}

Define the golden ratio conjugate as \(\Phi = \frac{\sqrt5-1}{2} \approx 0.61803\). After \(n\)
calls to \(f\), the space is reduced to about \(\Phi^n\) of its original size.

\begin{enumerate*}
\setcounter{enumi}{-1}
\item Should already know \((x_\mrm{L}, f_\mrm{L})\) and \((x_\mrm{R}, f_\mrm{R})\) from bounding
  algorithm.
\item Calculate the following:
  \begin{align}
    x_1 &= \Phi x_\mrm{L} + (1-\Phi) x_\mrm{R} \label{eq:golden-section-x1} \\
    f_1 &= f(x_1) \label{eq:golden-section-f1} \\
    x_2 &= (1-\Phi) x_\mrm{L} + \Phi x_\mrm{R} \label{eq:golden-section-x2} \\
    f_2 &= f(x_2) \label{eq:golden-section-f2}
  \end{align}
\item \label{itm:update-golden-section} If \(f_2 > f_1\), then
  \begin{itemize*}
  \item \(x_\mrm{R} := x_2\), \(f_\mrm{R} := f_2\)
  \item \(x_2 := x_1\), \(f_2 := f_1\)
  \item Compute the new values of \(x_1\), \(f_1\) using
    equations~\ref{eq:golden-section-x1}~and~\ref{eq:golden-section-f1}.
  \end{itemize*}
  else
  \begin{itemize*}
  \item \(x_\mrm{L} := x_1\), \(f_\mrm{L} := f_1\)
  \item \(x_1 := x_2\), \(f_1 := f_2\)
  \item Compute the new values of \(x_2\), \(f_2\) using
    equations~\ref{eq:golden-section-x2}~and~\ref{eq:golden-section-f2}.
  \end{itemize*}
\item Compute \(\epsilon = \frac{x_\mrm{R} - x_\mrm{L}}{L_0}\) and check convergence. If not
  converged, go to step~\ref{itm:update-golden-section}.
\end{enumerate*}

\subsubsection{Bisection Method}\label{sec:bisection-region}

This method uses first derivative information to eliminate half of the space on each iteration.

\begin{enumerate*}
\setcounter{enumi}{-1}
\item Should already know \((x_\mrm{L}, f_\mrm{L})\) and \((x_\mrm{R}, f_\mrm{R})\) from bounding
  algorithm. Also compute \(f'_\mrm{L} < 0\) and \(f'_\mrm{R} > 0\).
\item \label{itm:update-bisection-xm} Compute \(x_\mrm{m} = \frac{x_\mrm{L} + x_\mrm{R}}{2}\).
\item Compute \(f_\mrm{m} = f(x_\mrm{m})\) and \(f'_\mrm{m} = f'(x_\mrm{m})\).
\item If \(f'_\mrm{m} > 0\), then \(x_\mrm{R} := x_\mrm{m}\), else \(x_\mrm{L} := x_\mrm{m}\).
\item Compute \(\epsilon = \frac{x_\mrm{R} - x_\mrm{L}}{L_0}\) and check convergence. If not
  converged, go to step~\ref{itm:update-bisection-xm}.
\end{enumerate*}


\begin{thebibliography}{99}
\bibitem{flec} Ferguson, S. M., 2014, \emph{Engineering Design Optimization}, MAE 531 course
  lectures, North Carolina State University, Raleigh, NC.
\bibitem{eotap} Rao, S. S., 2009, \emph{Engineering Optimization: Theory and Practice}, 4th ed.,
  John Wiley \& Sons, Inc., Hoboken, NJ.
\end{thebibliography}

\end{document}
